\section{Conclusion}
With this paper, we have shown that the attitude estimation method using camera images and depth images leads to improved accuracy in unknown environments. The method of combining camera and depth images has been shown to be effective in segmentation tasks, but this paper confirms that it is also effective in the field of attitude estimation.

%本論文によって、カメラ画像と深度画像を用いたattitude推定手法はunknown環境において精度向上につながることを示した。これまでカメラ画像と深度画像を組み合わせる手法はセグメンテーションのタスクなどでは存在し有効性を示していたが、本論文によってそれがattitude推定の分野においても有効であることが確認された。


In this paper, a rather complex mechanism is introduced into the network to extract mutually interfering information from the depth and camera images, including the introduction of SA-Gate. Although it was able to achieve the desired effect, the complexity of the network led to a large increase in computational cost and processing time compared to the conventional method. Therefore, in order to put this method into practical use, it is necessary to improve it from various directions, such as revising the calculation method and enhancing the calculation capacity.

%本論文ではSA-Gateの導入など、深度画像とカメラ画像からの相互干渉し合う情報を抽出するためにやや複雑な機構をネットワークに導入した。それによって目的通りの効果を発揮することができたものの、ネットワークの複雑化は計算コストと処理時間の大きな増加を招き、従来手法と比べて処理時間は大きく増加した。したがって、本手法を実用化するにあたっては計算方法の見直しや計算能力の強化など、さまざまな方向からの改良が必要となる。

The proposed method demonstrated superior accuracy in attitude estimation in an unknown environment compared to conventional methods. In the future, we aim to develop a method that can perform more advanced attitude estimation for use with robots.

%提案手法はunknown environmentにおいて従来手法と比較して優れたattitude estimation accuracyを発揮した。今後はロボットへの搭載を前提としてより高度なattitude estimationを行える手法の開発発展を目指す。
